\documentclass[a4paper,11pt]{article}
\usepackage[utf8]{inputenc}
\usepackage{graphicx}
\usepackage[francais]{babel}
\usepackage{color}
\usepackage{epigraph}
\usepackage{hyperref}

\setlength\epigraphwidth{12cm}

\begin{document}

\title{{\textbf{Structures de données et algorithmes}\\}{Rapport de projet\\}
{\vspace{1cm}}{\includegraphics[width=120px]{uds-NoirBlanc.png}}{\vspace{3cm}
}}
\date{Semestre Printemps 2015}
\author{MEYER Cyril, ZANZI Zachary}
\maketitle 
\textbf{{Sujet :}}
Le projet de structures de données et algorithmes de ce semestre nous proposais de travailler avec des objets de type polyèdres.
\thispagestyle{empty}

\newpage
\setcounter{page}{1}

\section{Structures}
Liste et définition des structures
\\
\\
\textbf{{Nom :}} ListPoints
\\
\textbf{{Champs :}} int length; sommet *ldata;
\\
\\
\textbf{{Nom :}} ListArete
\\
\textbf{{Champs :}} int length; arete *ldata;
\\
\\
\textbf{{Nom :}} ListFaces
\\
\textbf{{Champs :}} int length; face *ldata;
\\
\\
\textbf{{Nom :}} ListInt
\\
\textbf{{Champs :}} int length; int *ldata;
\\
\\
\textbf{{Nom :}} polyedre , *pPolyedre
\\
\textbf{{Champs :}} ListPoints *S; ListArete *A; ListFaces *F;
\\
\\
\textbf{{Nom :}} sommet, *pSommet
\\
\textbf{{Champs :}} float x; float y; float z;
\\
\\
\textbf{{Nom :}} face, *pFace
\\
\textbf{{Champs :}} unsigned int idebut; unsigned int nbrearete;
\\
\\
\textbf{{Nom :}} listc, *pListc
\\
\textbf{{Champs :}} int length; pMaillon chaine;
\\
\\
\textbf{{Nom :}} maillon, *pMaillon
\\
\textbf{{Champs :}} int data; pMaillon s;
\\
\\
\newpage

\section{Fonctions / Spécifications}
Liste et définition des fonctions les plus importantes
\\
\\
\textbf{{Synopsis :}} retourne les faces de p contenant le sommet i
\\
\textbf{{Nom :}} FaceIncidentes
\\
\textbf{{Arguments :}} polyedre p, int i
\\
\textbf{{Type retour :}} ListInt
\\
\textbf{{Pré-condition :}} i ${\in}$ p
\\
\\
\textbf{{Synopsis :}} retourne les faces de p adjacentes au sommet i
\\
\textbf{{Nom :}} FaceAdjacentes
\\
\textbf{{Arguments :}} polyedre p, int i
\\
\textbf{{Type retour :}} ListInt
\\
\textbf{{Pré-condition :}} i ${\in}$ p
\\
\\
\textbf{{Synopsis :}} retourne les faces de p connecté au sommet i
\\
\textbf{{Nom :}} FaceConnectees
\\
\textbf{{Arguments :}} polyedre p, int i
\\
\textbf{{Type retour :}} ListInt
\\
\textbf{{Pré-condition :}} i ${\in}$ p
\\
\\
\newpage
\section{Conclusion}
Ce projet fût très instructif, il nous a permis de nous familiariser avec les structures de données complexe, ainsi que de travailler en équipe de façon efficace.
\subsection{Difficultés rencontrées}
La partie optimisation (listes chaînées) nous a posez beaucoup de problème d'implémentations.
\subsection{Amélioration et application possible}
\begin{itemize}
\item Modification et enregistrements du modèle sur un nouveau fichier
\item Modification mathématique du modèle
\begin{itemize}
\item Translation
\item Rotation
\item Symétrie
\item Projection
\item Homothétie
\end{itemize}
\item Appliquer une fonction sur l'ensemble des points (système de filtres en 3 dimension)
\end{itemize}


\newpage

\end{document}  